\chapter{Wie ist Cloud Computing zu bewerten?} % (fold)
\label{cha:Wie ist Cloud Computing zu bewerten?}

\section{Vorteile} % (fold)
\label{sec:Vorteile}

Der unumstrittene größte Vorteil ist mit Sicherheit der Kostenaspekt. Mit dem Wachstum des digitalen Markts geht eine regelrechte Datenexplosion einher, die großen Einfluss auf die Kosten und den Aufwand rund um Rechenleistung und vor allem auch Speicherkapazitäten hat. Gerade für kleine bis mittelständige Unternehmen kann eine Cloud-basierte Lösung hier Abhilfe schaffen, gerade weil die Kosten hier häufig nach Nutzungsdauer und/oder benötigten Ressourcen berechnet wird. \footcite[Vgl.][]{o.V..27.06.2022} \\
Außerdem schafft es Flexibilität. Die Firma kann Kosten und Zeit, die ansonsten für die Verwaltung der IT-Infrastruktur aufgebracht werden musste in andere Dinge investieren. Vor allem auch, weil sie sich nicht selbst um die Instandhaltung und vor allem Upgrades für das bestehende System kümmern muss. Das ist Aufgabe des Cloud-Providers. \footcite[Vgl.][]{o.V..27.06.2022} \\
Als weiterer, aktuell schnell und stark an Relevanz gewinnender Aspekt ist die Nachhaltigkeit zu nennen. Cloud-Infrastrukturen sind energietechnisch deutlich effizienter als On-Premises Infrastruktur. \footcite[Vgl.][S. 8]{Pohlmann.2022}

% section VorteileVorteile (end)

\section{Nachteile} % (fold)
\label{sec:Nachteile}

Das größte Risiko und damit den Hauptgrund warum Unternehmen keine Cloud-basierten Dienste nutzen ist nach wie vor die fragwürdige Sicherheit der Daten. Der ständige Datentransfer zwischen Cloud-Anwendung und Client bietet eine große Angriffsfläche. Auch die enormen Datenmengen, die ein solcher Cloud-Provider verwaltet sind natürlich sehr interessant für Angriffe. 43\% der Cloud-nutzenden Unternehmen in Österreich wurden in den vergangenen 12 Monaten (21/22) Ransomware-Attacken ausgesetzt und nur ungefähr zwei Drittel davon konnten mithilfe von Cloud-Security-Maßnahmen reduziert werden. \footcite[Vgl.][S. 5]{KPMGAGWirtschaftsprufungsgesellschaft.25.06.2022} \\
Als weiteres großes Risiko wird die Macht und Kontrolle marktführender Unternehmen wie Google betrachtet. Es ist schwer abzuschätzen, ob und in welchem Umfang Cloud-Anbieter während der Verarbeitung der Daten unautorisiert auf diese zugreifen. Außerdem wird eine sehr große Abhängigkeit vom Cloud-Anbieter geschaffen, da Schnittstellen zur Cloud oft herstellerspezifisch sind. \footcite[Vgl.][]{o.V..2022} \\
Neben den technischen Nachteilen stellen sich auch rechtliche Fragen, die bisher ungeklärt sind. Auf diese soll in diesem Rahmen allerdings nicht weiter eingegangen werden. 

% section Nachteile (end)

% chapter (end)
