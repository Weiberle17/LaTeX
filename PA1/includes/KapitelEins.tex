\chapter{Einleitung} % (fold)
\label{cha:Einleitung}

Durch die Corona-Pandemie und die damit verbundene Entwicklung möglichst viel im Homeoffice zu arbeiten wurde die Verwendung von VPN-Verbindungen für uns bereits zum Alltag. Zudem kommt die sich ankündigende Energiekrise. Da ein Anstieg der Energiekosten um 50 Prozent aktuell als realistisch betrachtet wird und von 30.000 Menschen gesprochen wird, die noch dieses Jahr arbeitslos werden könnten, scheint es wichtiger denn je zu sein Energiekosten zu sparen. \footcite[Vgl.][]{Gries.2022} Energieexperten berufen sich auf Studien, die eine Einsparung von bis zu fünf Prozent durch Homeoffice vorhersagen. \footcite[Vgl.][S. 1 f.]{Thaler.2022} Des Weiteren haben sich Homeoffice und flexible Arbeitszeiten zu starken Kriterien für die Berufswahl entwickelt. Besonders junge Menschen legen großen Wert darauf. Millenials würden Umfragen zu Folge doppelt so häufig den Arbeitsplatz wechseln als Babyboomer, wenn sie mit der zeitlichen oder örtlichen Flexibilität nicht zufrieden sind. \footcite[Vgl.][S. 1 f.]{Hahn.2021}

Für die Firma Komm.One, als regionales Dienstleistungsunternehmen, die das Rechenzentrum von Baden-Württemberg betreut und für einen möglichst reibungslosen Ablauf in den Rathäusern und Kommunen des Landes sorgt, ist es essentiell eine zuverlässige und beständige Arbeit zu leisten. Das ist nur möglich, wenn die Belegschaft durchgängig arbeiten kann. In den letzten zwei Jahren und vor allem zu den Corona-Hochzeiten konnte dies an den Standorten nicht gewährleistet werden, da bei der Infektion einer Person alle Kontaktperson in Quarantäne mussten. Als IT Firma blieb also die Möglichkeit des Homeoffice als sinnvolle Alternative. In dieser Zeit hat sich gezeigt, dass Homeoffice sowohl Vor- als auch Nachteile gegenüber dem zuvor standardmäßigen Arbeiten vor Ort hat. Je nach Aufgabe und Verantwortung ist dies natürlich unterschiedlich. Für die Komm.One hat sich das Homeoffice als wertvolle und auch in Zukunft interessante Alternative bestätigt. Aktuell haben allein am Standort Reutlingen etwa 180 von 250 Mitarbeitern einen Homeoffice-Vertrag, der bis zu fünf Homeoffice-Tage ermöglicht.

Um diese Menge an Homeoffice möglich zu machen, ist bei der Komm.One eine VPN-Infrastruktur gewachsen, die aus historischen Gründen, auf die in der Arbeit genauer eingegangen wird, nicht optimal ist und überarbeitet werden muss. Es werden nur für interne Mitarbeiter aktuell drei verschiedene VPN-Lösungen verwendet. Deshalb verfolgt die Komm.One das Ziel einer Harmonisierung der internen VPN-Struktur durchzuführen. In dieser Arbeit sollen die möglichen VPN-Lösungen evaluiert und die für die Komm.One beste Lösung gefunden werden.

% chapter (end)
