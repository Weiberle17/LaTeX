\chapter{Was ist ein Chatbot?} % (fold)
\label{cha:Was ist ein Chatbot?}

\section{Definition nach IBM} % (fold)
\label{sec:Definition nach IBM}

„Ein Chatbot ist eine Anwendung, die Künstliche Intelligenz verwendet, um sich mit Menschen in natürlicher Sprache zu unterhalten. Benutzer können Fragen stellen, auf welche das System in natürlicher Sprache antwortet. Er kann Texteingabe, Audioeingabe oder beides unterstützen.“ \footcite[][]{.2021}

% section Definition nach IBMDefinition nach IBM (end)

\section{Grundlegende Konzepte der Unterhaltungs-KI} % (fold)
\label{sec:Grundlegende Konzepte der Unterhaltungs-KI}

Die meisten funktionierenden Unterhaltungs-KIs beinhalten natürliche Sprachverarbeitung, Textanalytik, Text zu Sprache und Sprache zu Text-Umwandlungen. Außerdem wird meistens ein Text- bzw. Sprachenübersetzer verwendet. \footcite[][S. 2]{Bisser.2021}

% section Grundlegende Konzepte der Unterhaltungs-KI (end)

\subsection{Natürliche Sprachverarbeitung} % (fold)
\label{sub:Natürliche Sprachverarbeitung}

Mit natürlicher Sprachverarbeitung geht es um das Verstehen des Programmes oder der App von normalen Sätzen und das damit zusammenhängende Verbinden von ganzen Sätzen zu Aufgaben, die erledigt werden müssen. Beispielsweise werden die Sätze:

\begin{itemize}
	\item \glqq Was haben wir für ein Wetter? \grqq
	\item \glqq Wie ist das Wetter? \grqq
	\item \glqq Sag mir wie das Wetter ist \grqq
	\item \glqq Was sagt die Wettervorhersage? \grqq
	\item \glqq Wie warm ist es draußen? \grqq
\end{itemize}

im Grunde synonym verwendet. Egal welchen der Sätze der Nutzer im Chatbot verwendet will er, dass die App bzw. das Programm ihm das Wetter ausgibt. \\
Für das Programm ist also wichtig eine möglichst große Bandbreite an Sätzen zu erkennen und wie hier in dem Beispiel mit einer GetWeather Methode zu verbinden. Je mehr Formulierungen erkannt werden, desto benutzerfreundlicher ist das Programm. \footcite[][S. 3 ff.]{Bisser.2021}

% subsection Natürliche Sprachverarbeitung (end)

\subsection{Text zu Sprache und Sprache zu Text} % (fold)
\label{sub:Text zu Sprache und Sprache zu Text}

Grundsätzlich recht trivial ist die Wichtigkeit von Sprache zu Text und Text zu Sprache Funktionen. Ein Chatbot sollte sowohl mit Text- als auch mit Spracheingaben zurecht in gleichem Maße zurechtkommen. Ebenso sollte sowohl eine Text als auch eine Sprachausgabe möglich sein. Um diesen Anforderungen gerecht werden zu können sind die oben genannten Funktionen essentiell. \footcite[][S. 6 f.]{Bisser.2021}

% subsection Text zu Sprache und Sprache zu TextText zu Sprache und Sprache zu Text (end)

% chapter (end)
